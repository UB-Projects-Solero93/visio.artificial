\documentclass{article}
\usepackage[a4paper]{geometry}
\usepackage[utf8]{inputenc}

\usepackage{graphicx}
\graphicspath{ {./images_P2/imatges_informe/} }


\title{Informe pràctica 3\\Visió Artificial}
\author{Jordi Olivares Provencio\\Christian José Soler}

\begin{document}

\maketitle

\begin{enumerate}

 \item \textbf{Métodos de background substraction}

 \begin{enumerate}

 \item \textit{Encontrar  donde  se  acaba  una  escena  y  comienza  otra  (estos  frames  se denominan  shots del  vídeo).  ¿Qué  medida  de  las  imágenes  se  puede utilizar para distinguir las escenas?}
 
 A simple vista, no se pierden detalles, pero en el histograma se ve que hay diferencias sutiles entre las dos imágenes (original y reescalada).

 \item \textit{Aplicar  un  algoritmo  de  background  substraction  (consultar  el  material 
de teoría)}

 \item \textit{Visualizar, para  cada  segmento  delimitado  por  dos  shots  del  vídeo,  la  imagen 
estática extraída y las imágenes a partir de las cuales se ha obtenido}

 \item \textit{Comenta  en  detalle  la  implementación.  ¿Qué  sucede  si  los  shots  no  están 
correctamente  extraídos?    ¿Qué  sucede  si  encuentras  demasiados  shots  en  el 
vídeo?  Comenta qué representan  las  imágenes  estáticas  obtenidas.  ¿En  qué 
situaciones  el  algoritmo  funciona  y  en  cuáles  no? ¿Qué  sucede  si  restas  de  la 
imagen original la imagen del fondo? Visualízalo.}

 \item \textit{Ves  otras aplicaciones que se pueden sacar de este algoritmo?}

 \end{enumerate}

\newpage

 \item \textbf{Métodos de agrupación de datos numéricos}

 \begin{enumerate}
 \item \textit{La función gaussRandom (mu, sigma, numSamples), proporcionada con el 
enunciado,  permite  generar  nubes  de  puntos  con  una  distribución 
gaussiana con matriz de covarianza diagonal, utilizando como centro las 
coordenadas de mu. Genera tres nubes de 100 puntos con centros [1 2], [2 
2]  y  [2  1].  En  los  tres  casos  utilizar  una desviación  estándar  de  0.1  en 
todos los ejes. Visualiza los puntos generados (help: plot).}

 \item \textit{Utilizar el método kmeans para agrupar los datos anteriores. Visualizar en 
un mismo plot (help: subplot) una primera fila con los datos originales, y 
el  resultado  (incluyendo  los  centros)  de  las  agrupaciones  con  2,  3  y  4 
centros  respectivamente  en  la  segunda  fila  del  subplot,  utilizando 
diferentes colores (Figura 2).}

 \item \textit{Comenta  los  resultados  que  has  encontrado,  valorando  el  número  de 
agrupaciones que has encontrado en cada caso y la similitud con los datos 
iniciales}
 
 \end{enumerate}

\newpage

 \item \textbf{Métodos de agrupación: segmentación en el espacio RGB}

 \begin{enumerate}
 \item \textit{Lee  la  imagen   loro.png.  Conviértela  a  escala  de  grises  y  aplica  la 
segmentación  con  el  kmeans.  Prueba  diferentes  valores  de  k para 
encontrar la mejor segmentación.}

 \item \textit{Visualiza  la  imagen  segmentada  utilizando  el  nivel  de  gris  promedio 
encontrado con el método de segmentación. ¿A qué corresponde?}

 \item \textit{Añade como características las coordenadas de los píxeles y comprueba si 
mejora el resultado de la segmentación}

  \item \textit{Visualiza el resultado anterior en una  figura junto con la distribución de 
sus  colores  (utiliza  la  función plotPixelDistribution facilitada  con  el 
enunciado).}
  
  \item \textit{A partir de la imagen de entrada, crea un matriz que contenga en cada fila la tripleta RGB de un píxel de la imagen. Tendrá tantas filas como píxeles 
haya en la imagen.}

 \item \textit{Utilizar el método kmeans para reducir el número de colores de la imagen 
a 16 colores diferentes.}
 
 \item \textit{Visualizar  en  una  misma  figura  las  imágenes  del  primer  apartado,  y  la 
imagen con 16 colores junto con su distribución de colores (Figura 3).} 
 
 \end{enumerate}

\newpage

 \item \textbf{ Extracción de descriptores } 
 
 \begin{enumerate}
 \item \textit{¿A  qué  corresponden  las  variables f y  d que  devuelve  el método  vl\_sift? ¿Qué tamaño tienen? ¿A qué corresponden sus valores?}  
 
 \item \textit{En  este  apartado  se muestran  los  puntos  característicos  detectados  con SIFT  en  una  misma  imagen  con  rotación.  Compara  los  resultados 
obtenidos antes y después de hacer la rotación y comenta lo que ves. ¿Hay 
invariancia a rotación? ¿Qué significa la línea que aparece en el interior de 
los círculos?}

 \item \textit{En  este  caso,  se  comparan  los  resultados  con  una  misma  imagen  a 
distintas  escalas.  Compara  los  resultados  obtenidos  antes  y  después  de 
hacer el reescalado y comenta lo que ves. ¿Hay invariancia a escala? ¿Qué 
significa el tamaño de los círculos que se muestran?}

 \item \textit{En este apartado se genera una imagen sintética y se calcula el descriptor 
en dos puntos distintos de la imagen.  Verás  que  se generan dos  figuras 
con el mismo  formato, pero con una pequeña diferencia en el cálculo del 
descriptor.  Fíjate  en los  descriptores  de la  fila inferior  del subplot.  ¿Qué 
diferencia  encuentras  entre  los  mostrados  en  la  primera  figura  y  la 
segunda?}

 \end{enumerate}

\newpage

 \item \textbf{ Reconocimiento por alineación de puntos característicos}

 \begin{enumerate}
 \item \textit{Utiliza  el  método  showMatches usando  como  modelo  la  imagen 
 starbucks.jpg y como (nueva) escena la imagen starbucks6.jpg.}

 \item \textit{Repite el experimento  usando otras imágenes  como modelo  y/o escena. 
Para cada imagen modelo, enseña el  resto de imágenes de Starbucks en orden de su semejanza con el modelo.}

 \item \textit{¿Qué pasa si se le pasa una imagen que no contiene el logo de Starbucks? 
¿Qué  método  propondrías  (sin  implementarlo)  para  definir la 
probabilidad que la imagen “escena” corresponde a la imagen “modelo”?}

 \item \textit {Repite el experimento 4 veces cambiando las escalas y orientaciones del 
modelo. Comenta tus observaciones.}
 
 \end{enumerate}

\end{enumerate}

\end{document}
