\documentclass[a4paper,10pt]{article}
\usepackage[utf8]{inputenc}

%opening
\title{Cojonudo}
\author{Jordi Polivares}

\begin{document}

\maketitle
 
\begin{enumerate}
 
 \item Manipulación básica de imágenes
 
 \begin{enumerate}
 
  \item Escribir una  función (ej11()), que lee la imagen lena.jpg y la visualiza (figure, 
imshow).  ¿Qué  tamaño  tiene  la  imagen?  Cuántos  canales tiene? Convertir la 
imagen en escala de grises (Help: rgb2gray) (Figura 1a). 
 
 Tamaño : 225 x 225
 Canales : 3
 
 % Insertar imagen de lena en gris aquí
 
 \item Obtener información sobre su tamaño (Help: mirar el workspace y el comando 
size).  Calcular el 10\% de su tamaño.
 
 Se ve en el código
 
 \item “Incrustar”  la  imagen original  en una  imagen  negra con  un  borde  negro  de 
10\%  de  anchura/altura  de  la  imagen  original.  ¿De  qué  tamaño  ha  de  ser  la 
imagen negra?

 % \item  Mostrar la imagen resultante y guardarla como “lena_frame.jpg”.
 
 \end{enumerate}

 \item Cambios de color y contraste
 
 \begin{enumerate}
  \item Crear la función ej12() que abre la imagen y la aclara
de  forma  constante  sin  saturarla,  de  forma  que  el  píxel  más  claro  sea 
blanco (Figura 3 a)). Visualiza y 
compara el histograma  (help: imhist()) de las dos imágenes  y explica en 
qué se diferencian y en qué se parecen.

  Son el mismo histograma con un desplazamiento a la derecha el modificado

  \item Abre la imagen y la oscurece de  forma constante sin 
saturarla,  de  forma  que  el  píxel  más  oscuro  sea  negro  (Figura  3  b)). Visualiza y compara el histograma 
(help: imhist()) de las dos imágenes y explica en qué se diferencian y en 
qué se parecen.

  Son el mismo histograma con un desplazamiento a la derecho el modificado

  \item A  continuación  aumenta el  contraste  de  la  imagen. 
Primeramente, abre  la  imagen  y  la  convierte a  tipo  double  con  valores 
entre 0  y 1  (dividir  todos los  valores entre 255). Finalmente aumenta el 
contraste (Figura  3  c)). Visualiza y compara el histograma (help: imhist()) de las dos imágenes y 
explica en qué se diferencian y en qué se parecen.

  Son el mismo histograma, pero el modificado está extendido, para que ocupe todo el espectro.
  
 \end{enumerate}

 \item  Binarización de imágenes
 
 \begin{enumerate}
  \item Crea la versión binaria de la imagen original poniendo un umbral de 
valor  130 y  usando  la  indexación  lógica.  ¿Qué  pasa  si  utilizamos  un 
valor de umbral diferente?¿Por qué? Modificar la función que se pueda 
llamar con diferentes umbrales

  Las partes no blancas, a medida que subimos el umbral, más partes quedan absorvidas directamente al color negro. Lo mismo al revés.

  \item Visualiza las dos imágenes en la misma figura
  
  \item (opcional) Encuentra  el  comando  de  binarización  (thresholding)  en  Matlab  y 
reimplementa con este comando la  función ej13(). Help: puedes usar 
el lookfor.


 \end{enumerate}

 \item Trabajar con imágenes binarias y de escalas de gris 
 
 \begin{enumerate}
  \item  Abrir la imagen circles.bmp, visualizarla y convertirla a escala de grises (Figura 
6. a). Con imtool(im) puedes ver el valor de los píxeles de los círculos.
  
  \item Crear  3  imágenes  tipo  binario a  partir  de  la  imagen  inicial.  En  cada  una  de 
ellas, se debe representar el  fondo negro y cada uno de los 3 círculos en blanco 
individualmente,  utilizando  los  umbrales  de  valor  correspondiente  a  los  tres 
círculos (Figura 6. b). Visualizar las  tres imágenes en la misma  figura usando el 
subplot.

  \item Utiliza  las  matrices  binarias  creadas  en  b)  para  generar  3  imágenes,  donde 
aparezcan cada uno de los círculos con su nivel de gris original. (Figura 6. c)

  \item Combina las  3 imágenes  de  forma  que  se  obtenga la imagen mostrada en la 
Figura 6. c y visualízalos
  
 \end{enumerate}

 \item Visualización de imágenes en color 
 
 \begin{enumerate}
  \item 
 \end{enumerate}

 \item Creación de imágenes de 3 canales (en color)
 
 \begin{enumerate}
  \item Crea las 3 imágenes en escala de grises mostradas en la figura 8 (arriba).
  \item Combina las 3 imágenes de forma que se obtenga la imagen mostrada en 
la figura 8 (abajo).
  \item Guarda la imagen resultante como 3channels.jpg
 \end{enumerate}

 \item  Tratamiento de imágenes en color RGB: modificación de una imagen 
 
 \begin{enumerate}
  \item Leer la imagen logo.png
  \item Encontrar el vector de todos los pixeles que tiene color verde (6,118,85). Para 
ello utilizar la función find.
  \item Asignar a todos estos pixeles color rojo (255,0,0)
  \item Visualizar las dos imágenes en una figura de la forma siguiente:
 \end{enumerate}

 \item Tratamiento de imágenes en color RGB
 
 \begin{enumerate}
  \item Abrir  los  2  archivos  en  MATLAB.  ¿Cuántas  dimensiones  tienen  las 
imágenes?

  coat.png - 1188 x 915 x 3
  
  model.png - 1188 x 915 x 3
  
  \item Convertir la imagen coat.png en escala de grises utilizando la  función de 
MATLAB apropiada.

  \item Realizar una binarización sobre la imagen resultante en b) para conseguir 
2  regiones:  una  perteneciente  al  abrigo (foreground)  y  la  otra  al  fondo 
(background).  Crear  otra  imagen  binaria  invirtiendo  las  zonas  de 
foreground y background.

  \item Utilizar  las  matrices  binarias  creadas  en  c)  para  fusionar  las  imágenes 
model y  coat (fig. 11 c).

 \end{enumerate} 
 
\end{enumerate}


% \begin{abstract}
% 
% \end{abstract}

\end{document}
